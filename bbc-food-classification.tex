\documentclass[11pt,a4paper]{article}

\title{Decision Tree Classification of Cuisines for BBC Food Recipes}
\author{Ross Fenning}

\begin{document}

\maketitle

\section{Introduction}

Most recipes in modern, British cooking have
originated in several different cultures and cuisines, such as Indian,
Italian or French. Some, like British and French, have arguably a lot
of crossover, whilst some are much more further apart.

Other cuisines might seem superficially different to people, such as Indian and
South-east Asian cuisines, but perhaps they overlap in a lot of their
ingredients due to relative geographical proximity?


\subsection{The Problem}

This coursework will explore applying supervised machine learning
techniques on food recipes and within it, we hope to create a model for
identifying the cultural origin of a recipe from the ingredients used
therein.

The BBC Food website provides a searchable collection of around 15,000
recipes created by professional TV chefs -- spanning twenty different
cuisines -- all of which comprise ingredients, preparation steps and
further metadata such as suitability for specific diets (e.g.
vegetarian) and expected time required to prepare the meal.

However, only around a third of the recipes have actually been manually
categorised as to which cuisine they are. Marking the remaining
two-thirds with their respective cuisine would be further manual effort
by editorial staff. A model that is able to assign or at least suggest
a cuisine based on supervised learning from the known set could be a useful
tool in improving the metadata on BBC Food recipes.

\subsection{Attributes}

A recipe can be said to have the following attributes:

\begin{itemize}
  \item Ingredients
  \item Preparation steps
  \item Preparation time
  \item Cooking time
  \item Number of people it will serve
  \item Cuisine
  \item Special diets for which it may or may not be suitable
    (e.g. vegetarian, dairy-free)
  \item Whether it's in season
  \item For which course it is suitable (e.g. starter, dessert).
  \item Whether it pertains to an occasion (e.g. Burn's Night, Christmas)
  \item Which BBC programme has featured the recipe
  \item Which chef created the recipe
\end{itemize}

For simplicity, we will focus initially on classifying the cuisine class
attribute based on the ingredients expressed as vector of binary, nominal
attributes, i.e. the ``carrot'' attribute is \emph{true} if and only if the
given recipe contains carrots.

Whether a recipe is in season is a transient attribute, but could possibly be
used in a different problem of learning which recipes are in season in different
times of the year. The chef and TV programmes that featured the recipe would is
also out of scope for our problem, but could lead to an interesting machine
learning problem of trying to identify typical trends in recipes created by
particular chefs.

Occasion is categorical attribute that marks certain recipes as being typical
dishes served at particular festivals and celebrations. This could lead to
another classification problem, but for cuisine classification, it is likely to
reasonably redundant due to the cultural basis of such festivals. For instance,
there are unlikely to be many Italian Burn's Night dishes nor Mexican Diwali
recipes.

The numerical attributes such as number of people the recipe is to serve or the
number of steps in the cooking method seem fairly arbitrary with respect to the
ingredients involved or the cuisine. This indicates the problem is very much
dealing with categorical attributes.

\section{Analysis}

\section{Classification}
\subsection{Exploring the Dataset}
\subsection{Training a Cuisine Classifier}
\subsection{Evaluating the Model}
\subsection{Improving the Model}

\section{Discussion}
\subsection{Difficulties}
\subsection{Observations}
\subsection{Further Possibilties}

\end{document}
